\documentclass[catalan,varwidth=30em,border=15pt,crop=false,preview=true]{standalone}

% encoding
\usepackage[utf8]{inputenc}
\usepackage[T1]{fontenc}
\usepackage{lmodern}
\usepackage{babel}

% formatting and fixes
\usepackage{fixltx2e}
\frenchspacing
\usepackage[style=spanish]{csquotes}
\MakeAutoQuote{«}{»}

% general design preferences (page, paragraph indent/space, margins, class options, ...)
\setlength{\parskip}{20pt}
%%\pagestyle{plain}

% ADD ANY SPECIFIC PACKAGES HERE
% (CHEMISTRY, CODE, PUBLISHING)
\usepackage{circuitikz}
\usepackage{amsmath}

% other options

% hyperlink setup / metadata
\usepackage{hyperref}
\AfterPreamble{\hypersetup{
  pdfauthor={Xavier Mendez},
  pdftitle={Simplificació d'expressió algebraica},
  %%pdfsubject={},
}}

% document metadata
%%\author{}
%%\title{}
%%\date{}

\begin{document}


La expressió algebraica escrita directament del logigrama és:

\begin{equation*}
  f(a,b,c) = \overline{
    \overline{\left( a b c \right)} \cdot
    \overline{\left( a \overline{b} c \right)} \cdot
    \left( \overline{b} + c \right)
  }
\end{equation*}

Simplifiquem:

\begin{minipage}{\linewidth}
\begin{align*}
  f(a,b,c) &=
    \left( a b c \right) +
    \left( a \overline{b} c \right) +
    \overline{\left( \overline{b} + c \right)}
\\
  &=
    a b c +
    a \overline{b} c +
    b \overline{c}
\\
  &=
    a \left( b + \overline{b} \right) c +
    b \overline{c}
\\
  &=
    a c +
    b \overline{c}
\end{align*}
\end{minipage}

Fent el logigrama, resulta:

\begin{center} \begin{circuitikz}[scale=1] \draw
  (4,3) node[and port](and1){}
  (1.7,0) node[not port](c_not){}
  (4,1) node[and port](and2){}
  (6,2) node[or port](or){}

  (0,4) node[anchor=east]{$a$} -| (and1.in 1)
  (0,2) node[anchor=east]{$b$} -| (and2.in 1)
  (0,0) node[anchor=east]{$c$} -| (c_not.in)
  (c_not.out) -| (and2.in 2)
  (0,0) to[short, -*] ++(0.6,0) |- (and1.in 2)

  (and1.out) -| (or.in 1)
  (and2.out) -| (or.in 2)
  (or.out) -- (7,2) node[anchor=west]{$f$}

; \end{circuitikz} \end{center}


\end{document}
