\documentclass[catalan,border=15pt]{standalone}

% encoding
\usepackage[utf8]{inputenc}
\usepackage[T1]{fontenc}
\usepackage{lmodern}
\usepackage{babel}

% formatting and fixes
\usepackage{fixltx2e}
\frenchspacing
\usepackage[style=spanish]{csquotes}
\MakeAutoQuote{«}{»}

% ADD ANY SPECIFIC PACKAGES HERE
% (CHEMISTRY, CODE, PUBLISHING)
\usepackage{circuitikz}
\usepackage{amsmath}

% other options

% hyperlink setup / metadata
\usepackage{hyperref}
\AfterPreamble{\hypersetup{
  pdfauthor={Xavier Mendez},
  pdftitle={Simplificació d'expressió algebraica},
  pdfsubject={DGD},
}}

\begin{document}
\begin{minipage}{30em}
\setlength{\parskip}{7pt}


La expressió algebraica escrita directament del logigrama és:

\begin{equation*}
  f(a,b,c) = \overline{
    \overline{\left(\vphantom{\overline{b}} a b c \right)} \cdot
    \overline{\left( a \overline{b} c \right)} \cdot
    \left( \overline{b} + c \right)
  }
\end{equation*}

Simplifiquem:

\begin{align*}
  f(a,b,c) &= \overline{
    \overline{\left(\vphantom{\overline{b}} a b c \right)} \cdot
    \overline{\left( a \overline{b} c \right)} \cdot
    \left( \overline{b} + c \right)
  }
\\
  &=
    \left( a b c \right) +
    \left( a \overline{b} c \right) +
    \overline{\left( \overline{b} + c \right)}
\\
  &=
    a b c +
    a \overline{b} c +
    b \overline{c}
\\
  &=
    a \left( b + \overline{b} \right) c +
    b \overline{c}
\\
  &=
    a c +
    b \overline{c}
\end{align*}

Fent el logigrama, resulta:

\begin{center} \begin{circuitikz}[scale=1] \draw
  (4,3) node[and port](and1){}
  (1.7,0) node[not port](c_not){}
  (4,1) node[and port](and2){}
  (6,2) node[or port](or){}

  (0,4) node[anchor=east]{$a$} -| (and1.in 1)
  (0,2) node[anchor=east]{$b$} -| (and2.in 1)
  (0,0) node[anchor=east]{$c$} -| (c_not.in)
  (c_not.out) -| (and2.in 2)
  (0,0) to[short, -*] ++(0.6,0) |- (and1.in 2)

  (and1.out) -| (or.in 1)
  (and2.out) -| (or.in 2)
  (or.out) -- (7,2) node[anchor=west]{$f$}

; \end{circuitikz} \end{center}

Per comprovar que el comportament és equivalent a l'enunciat,
fem la taula de veritat i veiem que coincideix.

\begin{center} \begin{tabular}{cc}
$cba$ & $f$ \\
\hline
000 & 0 \\
001 & 0 \\
010 & 1 \\
011 & 1 \\
100 & 0 \\
101 & 1 \\
110 & 0 \\
111 & 1
\end{tabular} \end{center}


\end{minipage}
\end{document}
