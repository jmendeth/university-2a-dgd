\documentclass[catalan,border=15pt,class=scrartcl,multi=minipage,parskip=half*]{standalone}

% encoding
\usepackage[utf8]{inputenc}
\usepackage[T1]{fontenc}
\usepackage{lmodern}
\usepackage{babel}

% formatting and fixes
\frenchspacing
\usepackage[style=spanish]{csquotes}
\MakeAutoQuote{«}{»}
\usepackage{bookmark}

% ADD ANY SPECIFIC PACKAGES HERE
% (CHEMISTRY, CODE, PUBLISHING)
\usepackage{mathtools}
\usepackage{circuitikz}
\usetikzlibrary{calc}
\usetikzlibrary{automata}
\usetikzlibrary{circuits.logic.US,circuits.logic.IEC}
\usetikzlibrary{circuits.logic.mux}
%\usepackage{karnaugh}

% other options
\setcounter{tocdepth}{6}
\setcounter{secnumdepth}{2}

% hyperlink setup / metadata
\usepackage{hyperref}
\AfterPreamble{\hypersetup{
  pdfauthor={Xavier Mendez},
  pdfsubject={DGD},
}}

% custom commands
\newcommand{\startpage}{\begin{minipage}{30em}}
\newcommand{\finishpage}{\end{minipage}}
\newcommand{\iopair}[2]{\( \left(#1\right) \rightarrow #2 \)}

%FIXME: don't use circ/to short, it's from circuitikz

\AfterPreamble{\hypersetup{
  pdftitle={T7. Implementació sobre PLA},
}}

\begin{document}
\startpage

\subparagraph{Apartat A}

Escrivim el mapa de Karnaugh de les tres funcions, $f_2$, $f_1$ i $f_0$ respectivament:

\begin{center} \begin{tabular}{lcccc}
\hspace{-.7em} \tikz[baseline=.1em]{\node at (.5em,.5em) {$x_1x_0$}; \node at (-.5em,-.5em) {$x_2$}; \draw (1em,-1em) -- (-1em,1em);}
  & 00 & 01 & 11 & 10 \\
0 &  0 &  1 &  0 &  1 \\
1 &  0 &  1 &  0 &  0 \\
\end{tabular} \end{center}

\begin{center} \begin{tabular}{lcccc}
\hspace{-.7em} \tikz[baseline=.1em]{\node at (.5em,.5em) {$x_1x_0$}; \node at (-.5em,-.5em) {$x_2$}; \draw (1em,-1em) -- (-1em,1em);}
  & 00 & 01 & 11 & 10 \\
0 &  0 &  1 &  0 &  1 \\
1 &  0 &  1 &  1 &  1 \\
\end{tabular} \end{center}

\begin{center} \begin{tabular}{lcccc}
\hspace{-.7em} \tikz[baseline=.1em]{\node at (.5em,.5em) {$x_1x_0$}; \node at (-.5em,-.5em) {$x_2$}; \draw (1em,-1em) -- (-1em,1em);}
  & 00 & 01 & 11 & 10 \\
0 &  1 &  1 &  0 &  1 \\
1 &  0 &  0 &  0 &  1 \\
\end{tabular} \end{center}

Escrivim les expressions simplificades en SdP, tractant de reutilitzar productes comuns per tal de minimitzar el nombre de productes únics:

\begin{align*}
  f_2(x_2, x_1, x_0) &= \overline{x_1} x_0 + \overline{x_2} x_1 \overline{x_0} \\
  f_1(x_2, x_1, x_0) &= \overline{x_1} x_0 + x_2 x_0 + x_1 \overline{x_0} \\
  f_0(x_2, x_1, x_0) &= \overline{x_2} \overline{x_1} + x_1 \overline{x_0}
\end{align*}

\subparagraph{Apartat B}

La PLA mínima on es pot encabir $F$ consta de 3~entrades, 3~sortides i 5~productes.

\subparagraph{Apartat C}

El logigrama quedaria així:

\begin{center} \begin{tikzpicture}[circuit logic US, circuit ee IEC] \draw

(0,2.4) coordinate (d) node[anchor=east] {$x_2$} (d) -- ++(6.7,0)
(d) ++(1,.5) node[not gate US, draw] (d_not) {}
(d_not.input) -| ++(-.3,-.5) node[contact] {}
(d) ++(6.7,.5) -- (d_not.output)

(0,1.2) coordinate (d) node[anchor=east] {$x_1$} (d) -- ++(6.7,0)
(d) ++(1,.5) node[not gate US, draw] (d_not) {}
(d_not.input) -| ++(-.3,-.5) node[contact] {}
(d) ++(6.7,.5) -- (d_not.output)

(0,0.0) coordinate (d) node[anchor=east] {$x_0$} (d) -- ++(6.7,0)
(d) ++(1,.5) node[not gate US, draw] (d_not) {}
(d_not.input) -| ++(-.3,-.5) node[contact] {}
(d) ++(6.7,.5) -- (d_not.output)


(2,-1) node[and gate, point down, logic gate inputs=nn] (and) {}
(and.west) -- ++(0,4)
(and.output) -- ++(0,-3)
(2,1.7) node[contact] {}
(2,0.0) node[contact] {}

(3,-1) node[and gate, point down, logic gate inputs=nn] (and) {}
(and.west) -- ++(0,4)
(and.output) -- ++(0,-3)
(3,2.4) node[contact] {}
(3,0.0) node[contact] {}

(4,-1) node[and gate, point down, logic gate inputs=nn] (and) {}
(and.west) -- ++(0,4)
(and.output) -- ++(0,-3)
(4,1.2) node[contact] {}
(4,0.5) node[contact] {}

(5,-1) node[and gate, point down, logic gate inputs=nn] (and) {}
(and.west) -- ++(0,4)
(and.output) -- ++(0,-3)
(5,2.9) node[contact] {}
(5,1.7) node[contact] {}

(6,-1) node[and gate, point down, logic gate inputs=nn] (and) {}
(and.west) -- ++(0,4)
(and.output) -- ++(0,-3)
(6,2.9) node[contact] {}
(6,1.2) node[contact] {}
(6,0.5) node[contact] {}


(7,-2.2) node[or gate, logic gate inputs=nn] (or) {}
(or.west) -- ++(-5.3,0)
(or.output) -- ++(.5,0) node[anchor=west] {$f_2$}
(2,-2.2) node[contact] {}
(6,-2.2) node[contact] {}

(7,-3.0) node[or gate, logic gate inputs=nn] (or) {}
(or.west) -- ++(-5.3,0)
(or.output) -- ++(.5,0) node[anchor=west] {$f_1$}
(2,-3.0) node[contact] {}
(3,-3.0) node[contact] {}
(4,-3.0) node[contact] {}

(7,-3.8) node[or gate, logic gate inputs=nn] (or) {}
(or.west) -- ++(-5.3,0)
(or.output) -- ++(.5,0) node[anchor=west] {$f_0$}
(5,-3.8) node[contact] {}
(4,-3.8) node[contact] {}

; \end{tikzpicture} \end{center}

\finishpage
\end{document}
