\documentclass[catalan,border=15pt,class=scrartcl,multi=minipage,parskip=half*]{standalone}

% encoding
\usepackage[utf8]{inputenc}
\usepackage[T1]{fontenc}
\usepackage{lmodern}
\usepackage{babel}

% formatting and fixes
\frenchspacing
\usepackage[style=spanish]{csquotes}
\MakeAutoQuote{«}{»}
\usepackage{bookmark}

% ADD ANY SPECIFIC PACKAGES HERE
% (CHEMISTRY, CODE, PUBLISHING)
\usepackage{mathtools}
\usepackage{circuitikz}
\usetikzlibrary{calc}
\usetikzlibrary{automata}
\usetikzlibrary{circuits.ee.IEC}
\usetikzlibrary{circuits.logic.US,circuits.logic.IEC}
\usetikzlibrary{circuits.logic.mux}
%\usepackage{karnaugh}
\usepackage{minted}

% other options
\setcounter{tocdepth}{6}
\setcounter{secnumdepth}{2}

% hyperlink setup / metadata
\usepackage{hyperref}
\AfterPreamble{\hypersetup{
  pdfauthor={Xavier Mendez},
  pdfsubject={DGD},
}}

% custom commands
\newcommand{\startpage}{\begin{minipage}{30em} \setlength{\parskip}{0.5em}}
\newcommand{\finishpage}{\end{minipage}}
\newcommand{\iopair}[2]{\( \left(#1\right) \rightarrow #2 \)}

\renewcommand{\startpage}{\begin{minipage}{40em} \setlength{\parskip}{0.5em}}
\AfterPreamble{\hypersetup{
  pdftitle={T12. Càlcul de tensió V\_OL},
}}

\begin{document}
\startpage

Anem a analitzar, primer, l'inversor sense donar valors a cap paràmetre. Modelem el transistor com:
%
\begin{equation*}
  I_D = \begin{cases}
    0 & \text{si $V_e \leq 0$} \\
    K \left( V_{DS}V_e - \frac{V_{DS}^2}{2} \right) & \text{si $V_e > 0; V_{DS} < V_e$} \\
    K \frac{V_e^2}{2} & \text{si $V_e > 0; V_{DS} \geq V_e$}
  \end{cases}
\end{equation*}

On $K > 0$, $V_{DS} \geq 0$ i $V_e = V_{GS} - V_T$.

Apliquem el model al circuit
i escrivim KCLs per a tots els nodes desconeguts:
%
\begin{align*}
\begin{cases}
  I_D = \begin{cases}
    0 & \text{si $V_e \leq 0$} \\
    K \left( V_oV_e - \frac{V_o^2}{2} \right) & \text{si $V_e > 0; V_o < V_e$} \\
    K \frac{V_e^2}{2} & \text{si $V_e > 0; V_o \geq V_e$}
  \end{cases} \\
%
  V_e = V_i - V_T \\
%
  I_D = \frac{V_{DD} - V_o}{R_L} \\
%
  R_L, K, V_{DD} > 0
\end{cases}
\end{align*}

Ara particularitzem per a cada tram.

\finishpage
\startpage


\paragraph{Zona de tall}

\begin{align*}
  I_D &= 0 \\
  V_e &\leq 0
\end{align*}

Es veu a simple vista que si substituim $I_D$ en el KCL, resulta:
%
\begin{align*}
  V_o = V_{DD}
\end{align*}


\paragraph{Zona de saturació}

\begin{align*}
  I_D &= K \frac{V_e^2}{2} \\
  V_e &> 0 \\
  V_o &\geq V_e
\end{align*}

Subsituïm $I_D$ al KCL i aïllem $V_o$ en funció de $V_e$:
%
\begin{align*}
  K \frac{V_e^2}{2} &= \frac{V_{DD} - V_o}{R_L} \\
  V_o &= V_{DD} - \frac{R_L K}{2} V_e^2
\end{align*}

Ara substituïm $V_o$ en la condició $V_o \geq V_e$ per expressar-la en funció
de $V_e$:
%
\begin{align*}
  V_{DD} - \frac{R_L K}{2} V_e^2 &\geq V_e \\
  \frac{R_L K}{2} V_e^2 + V_e - V_{DD} &\leq 0
\end{align*}


\paragraph{Zona òhmica}

\begin{align*}
  I_D &= K \left( V_oV_e - \frac{V_o^2}{2} \right) \\
  V_e &> 0 \\
  V_o &< V_e
\end{align*}

Substituïm $I_D$ al KCL i aïllem $V_o$ en funció de $V_e$:
%
\begin{align*}
  K \left( V_oV_e - \frac{V_o^2}{2} \right) &= \frac{V_{DD} - V_o}{R_L} \\
  V_oV_e - \frac{V_o^2}{2} &= \frac{V_{DD}}{R_L K} - \frac{V_o}{R_L K} \\
  0 &= \frac{V_o^2}{2} - V_oV_e - \frac{V_o}{R_L K} + \frac{V_{DD}}{R_L K} \\
  0 &= \frac{R_L K}{2} V_o^2 - \left(R_L K V_e + 1\right) V_o + V_{DD} \\
  V_o &= \frac{\left(R_L K V_e + 1\right) \pm \sqrt{\left(R_L K V_e + 1\right)^2 - 2 R_L K V_{DD}}}{R_L K} \\
  V_o &= V_e + \frac{1 \pm \sqrt{\left(R_L K V_e + 1\right)^2 - 2 R_L K V_{DD}}}{R_L K}
\end{align*}
%
Per a satisfer $V_o < V_e$ la fracció ha de ser negativa. Per tant, podem
descartar una de les solucions:
%
\begin{align*}
  V_o &= V_e + \frac{1 - \sqrt{\left(R_L K V_e + 1\right)^2 - 2 R_L K V_{DD}}}{R_L K}
\end{align*}
%
En particular, perquè la fracció sigui negativa, el radicand ha de ser més gran
que la unitat. Això garantitza que $V_o$ existeix i satisfà $V_o < V_e$.
%
\begin{align*}
  \left(R_L K V_e + 1\right)^2 - 2 R_L K V_{DD} &> 1 \\
  \frac{R_L K}{2} V_e^2 + V_e - V_{DD} &> 0
\end{align*}

\finishpage
\startpage


Ara podem escriure $V_o$ en funció de $V_e$, definit com a funció a trams:
%
\begin{equation*}
  V_o = \begin{cases}
    V_{DD} & \text{si $V_e \leq 0$} \\
    V_e + \frac{1 - \sqrt{\left(R_L K V_e + 1\right)^2 - 2 R_L K V_{DD}}}{R_L K} & \text{si $V_e > 0; \frac{R_L K}{2} V_e^2 + V_e - V_{DD} > 0$} \\
    V_{DD} - \frac{R_L K}{2} V_e^2 & \text{si $V_e > 0; \frac{R_L K}{2} V_e^2 + V_e - V_{DD} \leq 0$}
  \end{cases}
\end{equation*}

Simplificarem ara les condicions complementàries del segon i tercer tram.
%
\begin{align*}
  \frac{R_L K}{2} V_e^2 + V_e - V_{DD} &\leq 0
\end{align*}
%
El disciminant del polinomi de segon grau és $1 + 2 R_L K V_{DD}$, que sempre
és positiu. Per tant, la condició és equivalent a:
%
\begin{align*}
  V_e &\in \left[ \frac{-\sqrt{1 + 2 R_L K V_{DD}} -1}{R_L K}, \frac{\sqrt{1 + 2 R_L K V_{DD}} -1}{R_L K} \right]
\end{align*}
%
Com que a més, $V_e > 0$ i l'extrem inferior de l'interval és negatiu, podem escriure simplement:
%
\begin{equation*}
  V_e \leq \frac{\sqrt{1 + 2 R_L K V_{DD}} -1}{R_L K}
\end{equation*}

Procedim de forma similar amb la condició del tercer tram, que és complementària:
%
\begin{align*}
  V_e &\notin \left[ \frac{-\sqrt{1 + 2 R_L K V_{DD}} -1}{R_L K}, \frac{\sqrt{1 + 2 R_L K V_{DD}} -1}{R_L K} \right]
\end{align*}
\begin{equation*}
  V_e > \frac{\sqrt{1 + 2 R_L K V_{DD}} -1}{R_L K}
\end{equation*}

La corba de transferència resultant, amb les condicions simplificades
i substituint $V_e = V_i - V_T$, resulta:

\begin{equation*}
  V_o = \begin{cases}
    V_{DD} & \text{si $V_i \leq V_T$} \\
    \left(V_i - V_T\right) + \frac{1 - \sqrt{\left(k \left(V_i - V_T\right) + 1\right)^2 - 2 k V_{DD}}}{k} & \text{si $V_T < V_i \leq V'_T$} \\
    V_{DD} - \frac{k}{2} \left(V_i - V_T\right)^2 & \text{si $V_i > V'_T$}
  \end{cases}
\end{equation*}
%
\begin{center}
amb $V'_T = V_T + \frac{\sqrt{1 + 2 k V_{DD}} -1}{k}$ i $k = R_L K$
\end{center}


\finishpage
\end{document}
