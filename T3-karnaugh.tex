\documentclass[catalan,border=15pt,class=scrartcl,multi=minipage,parskip=half*]{standalone}

% encoding
\usepackage[utf8]{inputenc}
\usepackage[T1]{fontenc}
\usepackage{lmodern}
\usepackage{babel}

% formatting and fixes
\frenchspacing
\usepackage[style=spanish]{csquotes}
\MakeAutoQuote{«}{»}
\usepackage{bookmark}

% ADD ANY SPECIFIC PACKAGES HERE
% (CHEMISTRY, CODE, PUBLISHING)
\usepackage{mathtools}
\usepackage{circuitikz}
\usetikzlibrary{calc}
\usetikzlibrary{automata}
\usetikzlibrary{circuits.logic.US,circuits.logic.IEC}
\usetikzlibrary{circuits.logic.mux}
%\usepackage{karnaugh}

% other options
\setcounter{tocdepth}{6}
\setcounter{secnumdepth}{2}

% hyperlink setup / metadata
\usepackage{hyperref}
\AfterPreamble{\hypersetup{
  pdfauthor={Xavier Mendez},
  pdfsubject={DGD},
}}

% custom commands
\newcommand{\startpage}{\begin{minipage}{30em}}
\newcommand{\finishpage}{\end{minipage}}
\newcommand{\iopair}[2]{\( \left(#1\right) \rightarrow #2 \)}

%FIXME: don't use circ/to short, it's from circuitikz

\AfterPreamble{\hypersetup{
  pdftitle={T3. Simplificació mitjançant Karnaugh},
}}

\newcommand{\D}{\overline{\vphantom{dbca}d}}
\newcommand{\C}{\overline{\vphantom{dbca}c}}
\newcommand{\B}{\overline{\vphantom{dbca}b}}
\newcommand{\A}{\overline{\vphantom{dbca}a}}

\begin{document}
\startpage


L'expressió donada és:
%
\begin{equation*}
  f(d,c,b,a) = m_1 + m_5 + m_7 + m_9 + m_{11}
\end{equation*}

O bé (suposant indexat $dcba$):
%
\begin{equation*}
  f(d,c,b,a) = \D \C \B a + \D c \B a + \D c b a + d \C \B a + d \C b a
\end{equation*}

Fem el mapa de Karnaugh:

\begin{center} \begin{tabular}{lcccc}
\hspace{-.7em} \tikz[baseline=.1em]{\node at (.5em,.5em) {$ba$}; \node at (-.5em,-.5em) {$dc$}; \draw (1em,-1em) -- (-1em,1em);}
   & 00 & 01 & 11 & 10 \\
00 &  0 &  1 &  0 &  0 \\
01 &  0 &  1 &  1 &  0 \\
11 &  0 &  0 &  0 &  0 \\
10 &  0 &  1 &  1 &  0 \\
\end{tabular} \end{center}


\subsubsection{Suma de productes}

L'expressió simplificada en SdP quedaria:
%
\begin{equation*}
  f(d,c,b,a) = \D \B a + \D c a + d \C a
\end{equation*}

S'implementaria amb les portes següents:

\begin{center} \begin{tabular}{rl}
3 & \textsf{NOT} \\
3 & \textsf{AND3} \\
1 & \textsf{OR3} \\
\end{tabular} \end{center}

El logigrama quedaria així:

\begin{center} \begin{tikzpicture}[circuit logic US] \draw

(0.0,0) coordinate (d) node[anchor=south] {$d$} (d) -- ++(0,-5)
(d) ++(.5, -1) node[not gate US, draw, point down] (d_not) {}
(d_not.input) -- ++(0,.3) to[short, -*] ++(-.5,0)
(d) ++(.5,-5) -- (d_not.output)

(1.2,0) coordinate (d) node[anchor=south] {$c$} (d) -- ++(0,-5)
(d) ++(.5, -1) node[not gate US, draw, point down] (d_not) {}
(d_not.input) -- ++(0,.3) to[short, -*] ++(-.5,0)
(d) ++(.5,-5) -- (d_not.output)

(2.4,0) coordinate (d) node[anchor=south] {$b$} (d) -- ++(0,-5)
(d) ++(.5, -1) node[not gate US, draw, point down] (d_not) {}
(d_not.input) -- ++(0,.3) to[short, -*] ++(-.5,0)
(d) ++(.5,-5) -- (d_not.output)

(3.6,0) coordinate (d) node[anchor=south] {$a$} (d) -- ++(0,-5)
%(d) ++(.5, -1) node[not gate US, draw, point down] (d_not) {}
%(d_not.input) -- ++(0,.3) to[short, -*] ++(-.5,0)
%(d) ++(.5,-5) -- (d_not.output)

;

\node at (5,-2) [and gate, inputs=nnn] (and1) {};
\foreach \a in {1,...,3}
  \draw (and1.input \a -| -0.5,0) -- (and1.input \a);

\node at (5,-3) [and gate, inputs=nnn] (and2) {};
\foreach \a in {1,...,3}
  \draw (and2.input \a -| -0.5,0) -- (and2.input \a);

\node at (5,-4) [and gate, inputs=nnn] (and3) {};
\foreach \a in {1,...,3}
  \draw (and3.input \a -| -0.5,0) -- (and3.input \a);

\draw
  (6.4,-3) node[or gate, inputs=nnn] (or) {}
  (or.input 1) -- ++(-0.2,0) |- (and1.output)
  (or.input 2) -- ++(-0.2,0) |- (and2.output)
  (or.input 3) -- ++(-0.2,0) |- (and3.output)
  (or.output) -- ++(0.5,0) node[anchor=west] {$f$}
;

\draw
  (and1.input 1 -| 0.0,0) ++(.5,0) node[circ] {}
  (and1.input 2 -| 2.4,0) ++(.5,0) node[circ] {}
  (and1.input 3 -| 3.6,0)          node[circ] {}

  (and2.input 1 -| 0.0,0) ++(.5,0) node[circ] {}
  (and2.input 2 -| 1.2,0)          node[circ] {}
  (and2.input 3 -| 3.6,0)          node[circ] {}

  (and3.input 1 -| 0.0,0)          node[circ] {}
  (and3.input 2 -| 1.2,0) ++(.5,0) node[circ] {}
  (and3.input 3 -| 3.6,0)          node[circ] {}
;

\end{tikzpicture} \end{center}

Passada a forma \textsf{NAND} de \textsf{NAND}s, l'expressió queda:
%
\begin{equation*}
  f(d,c,b,a) = \overline{
    \overline{\left( \D \B a \right)} \cdot
    \overline{\left( \D c a \right)} \cdot
    \overline{\left( d \C a \right)}
  }
\end{equation*}

El logigrama resultant és:

\begin{center} \begin{tikzpicture}[circuit logic US] \draw

(0.0,0) coordinate (d) node[anchor=south] {$d$} (d) -- ++(0,-5)
(d) ++(.5, -1) node[not gate US, draw, point down] (d_not) {}
(d_not.input) -- ++(0,.3) to[short, -*] ++(-.5,0)
(d) ++(.5,-5) -- (d_not.output)

(1.2,0) coordinate (d) node[anchor=south] {$c$} (d) -- ++(0,-5)
(d) ++(.5, -1) node[not gate US, draw, point down] (d_not) {}
(d_not.input) -- ++(0,.3) to[short, -*] ++(-.5,0)
(d) ++(.5,-5) -- (d_not.output)

(2.4,0) coordinate (d) node[anchor=south] {$b$} (d) -- ++(0,-5)
(d) ++(.5, -1) node[not gate US, draw, point down] (d_not) {}
(d_not.input) -- ++(0,.3) to[short, -*] ++(-.5,0)
(d) ++(.5,-5) -- (d_not.output)

(3.6,0) coordinate (d) node[anchor=south] {$a$} (d) -- ++(0,-5)
%(d) ++(.5, -1) node[not gate US, draw, point down] (d_not) {}
%(d_not.input) -- ++(0,.3) to[short, -*] ++(-.5,0)
%(d) ++(.5,-5) -- (d_not.output)

;

\node at (5,-2) [nand gate, inputs=nnn] (and1) {};
\foreach \a in {1,...,3}
  \draw (and1.input \a -| -0.5,0) -- (and1.input \a);

\node at (5,-3) [nand gate, inputs=nnn] (and2) {};
\foreach \a in {1,...,3}
  \draw (and2.input \a -| -0.5,0) -- (and2.input \a);

\node at (5,-4) [nand gate, inputs=nnn] (and3) {};
\foreach \a in {1,...,3}
  \draw (and3.input \a -| -0.5,0) -- (and3.input \a);

\draw
  (6.4,-3) node[nand gate, inputs=nnn] (or) {}
  (or.input 1) -- ++(-0.2,0) |- (and1.output)
  (or.input 2) -- ++(-0.2,0) |- (and2.output)
  (or.input 3) -- ++(-0.2,0) |- (and3.output)
  (or.output) -- ++(0.5,0) node[anchor=west] {$f$}
;

\draw
  (and1.input 1 -| 0.0,0) ++(.5,0) node[circ] {}
  (and1.input 2 -| 2.4,0) ++(.5,0) node[circ] {}
  (and1.input 3 -| 3.6,0)          node[circ] {}

  (and2.input 1 -| 0.0,0) ++(.5,0) node[circ] {}
  (and2.input 2 -| 1.2,0)          node[circ] {}
  (and2.input 3 -| 3.6,0)          node[circ] {}

  (and3.input 1 -| 0.0,0)          node[circ] {}
  (and3.input 2 -| 1.2,0) ++(.5,0) node[circ] {}
  (and3.input 3 -| 3.6,0)          node[circ] {}
;

\end{tikzpicture} \end{center}


\subsubsection{Producte de sumes}

L'expressió simplificada en PdS quedaria:
%
\begin{equation*}
  f(d,c,b,a) = \left( \D + \C \right) \cdot \left( a \right) \cdot \left( d + c + \B \right)
\end{equation*}

S'implementaria amb les portes següents:

\begin{center} \begin{tabular}{rl}
3 & \textsf{NOT} \\
1 & \textsf{OR2} \\
1 & \textsf{OR3} \\
1 & \textsf{AND3} \\
\end{tabular} \end{center}

El logigrama quedaria així:

\begin{center} \begin{tikzpicture}[circuit logic US] \draw

(0.0,0) coordinate (d) node[anchor=south] {$d$} (d) -- ++(0,-5)
(d) ++(.5, -1) node[not gate US, draw, point down] (d_not) {}
(d_not.input) -- ++(0,.3) to[short, -*] ++(-.5,0)
(d) ++(.5,-5) -- (d_not.output)

(1.2,0) coordinate (d) node[anchor=south] {$c$} (d) -- ++(0,-5)
(d) ++(.5, -1) node[not gate US, draw, point down] (d_not) {}
(d_not.input) -- ++(0,.3) to[short, -*] ++(-.5,0)
(d) ++(.5,-5) -- (d_not.output)

(2.4,0) coordinate (d) node[anchor=south] {$b$} (d) -- ++(0,-5)
(d) ++(.5, -1) node[not gate US, draw, point down] (d_not) {}
(d_not.input) -- ++(0,.3) to[short, -*] ++(-.5,0)
(d) ++(.5,-5) -- (d_not.output)

(3.6,0) coordinate (d) node[anchor=south] {$a$} (d) -- ++(0,-5)
%(d) ++(.5, -1) node[not gate US, draw, point down] (d_not) {}
%(d_not.input) -- ++(0,.3) to[short, -*] ++(-.5,0)
%(d) ++(.5,-5) -- (d_not.output)

;

\node at (5,-2) [or gate, inputs=nn] (and1) {};
\foreach \a in {1,...,2}
  \draw (and1.input \a -| -0.5,0) -- (and1.input \a);

\coordinate (and2) at (5,-3);
\draw (and2 -| -0.5,0) -- (and2);

\node at (5,-4) [or gate, inputs=nnn] (and3) {};
\foreach \a in {1,...,3}
  \draw (and3.input \a -| -0.5,0) -- (and3.input \a);

\draw
  (6.4,-3) node[and gate, inputs=nnn] (or) {}
  (or.input 1) -- ++(-0.2,0) |- (and1.output)
  (or.input 2) -- ++(-0.2,0) |- (and2)
  (or.input 3) -- ++(-0.2,0) |- (and3.output)
  (or.output) -- ++(0.5,0) node[anchor=west] {$f$}
;

\draw
  (and1.input 1 -| 0.0,0) ++(.5,0) node[circ] {}
  (and1.input 2 -| 1.2,0) ++(.5,0) node[circ] {}

  (and2         -| 3.6,0)          node[circ] {}

  (and3.input 1 -| 0.0,0)          node[circ] {}
  (and3.input 2 -| 1.2,0)          node[circ] {}
  (and3.input 3 -| 2.4,0) ++(.5,0) node[circ] {}
;

\end{tikzpicture} \end{center}

Passada a forma \textsf{NOR} de \textsf{NOR}s, l'expressió queda:
%
\begin{equation*}
  f(d,c,b,a) = \overline{
    \overline{\left( \D + \C \right)} +
    \overline{\left( a \right)} +
    \overline{\left( d + c + \B \right)}
  }
\end{equation*}

El logigrama resultant és:

\begin{center} \begin{tikzpicture}[circuit logic US] \draw

(0.0,0) coordinate (d) node[anchor=south] {$d$} (d) -- ++(0,-5)
(d) ++(.5, -1) node[not gate US, draw, point down] (d_not) {}
(d_not.input) -- ++(0,.3) to[short, -*] ++(-.5,0)
(d) ++(.5,-5) -- (d_not.output)

(1.2,0) coordinate (d) node[anchor=south] {$c$} (d) -- ++(0,-5)
(d) ++(.5, -1) node[not gate US, draw, point down] (d_not) {}
(d_not.input) -- ++(0,.3) to[short, -*] ++(-.5,0)
(d) ++(.5,-5) -- (d_not.output)

(2.4,0) coordinate (d) node[anchor=south] {$b$} (d) -- ++(0,-5)
(d) ++(.5, -1) node[not gate US, draw, point down] (d_not) {}
(d_not.input) -- ++(0,.3) to[short, -*] ++(-.5,0)
(d) ++(.5,-5) -- (d_not.output)

(3.6,0) coordinate (d) node[anchor=south] {$a$} (d) -- ++(0,-5)
(d) ++(.5, -1) node[not gate US, draw, point down] (d_not) {}
(d_not.input) -- ++(0,.3) to[short, -*] ++(-.5,0)
(d) ++(.5,-5) -- (d_not.output)

;

\node at (5,-2) [nor gate, inputs=nn] (and1) {};
\foreach \a in {1,...,2}
  \draw (and1.input \a -| -0.5,0) -- (and1.input \a);

\coordinate (and2) at (5,-3);
\draw (and2 -| -0.5,0) -- (and2);

\node at (5,-4) [nor gate, inputs=nnn] (and3) {};
\foreach \a in {1,...,3}
  \draw (and3.input \a -| -0.5,0) -- (and3.input \a);

\draw
  (6.4,-3) node[nor gate, inputs=nnn] (or) {}
  (or.input 1) -- ++(-0.2,0) |- (and1.output)
  (or.input 2) -- ++(-0.2,0) |- (and2)
  (or.input 3) -- ++(-0.2,0) |- (and3.output)
  (or.output) -- ++(0.5,0) node[anchor=west] {$f$}
;

\draw
  (and1.input 1 -| 0.0,0) ++(.5,0) node[circ] {}
  (and1.input 2 -| 1.2,0) ++(.5,0) node[circ] {}

  (and2         -| 3.6,0) ++(.5,0) node[circ] {}

  (and3.input 1 -| 0.0,0)          node[circ] {}
  (and3.input 2 -| 1.2,0)          node[circ] {}
  (and3.input 3 -| 2.4,0) ++(.5,0) node[circ] {}
;

\end{tikzpicture} \end{center}


\finishpage
\end{document}
