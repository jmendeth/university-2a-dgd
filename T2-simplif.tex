\documentclass[catalan,border=15pt]{standalone}

% encoding
\usepackage[utf8]{inputenc}
\usepackage[T1]{fontenc}
\usepackage{lmodern}
\usepackage{babel}

% formatting and fixes
\usepackage{fixltx2e}
\frenchspacing
\usepackage[style=spanish]{csquotes}
\MakeAutoQuote{«}{»}

% ADD ANY SPECIFIC PACKAGES HERE
% (CHEMISTRY, CODE, PUBLISHING)
\usepackage{mathtools}
\usepackage{circuitikz}
\usetikzlibrary{circuits.logic.US,circuits.logic.IEC}
\usetikzlibrary{calc}

% other options

% hyperlink setup / metadata
\usepackage{hyperref}
\AfterPreamble{\hypersetup{
  pdfauthor={Xavier Mendez},
  pdftitle={T2. Simplificació d'expressió algebraica},
  pdfsubject={DGD},
}}

\begin{document}
\begin{minipage}{30em}
\setlength{\parskip}{7pt}


La expressió algebraica escrita directament del logigrama és:

\begin{equation*}
  f(a,b,c) = \overline{
    \overline{\left(\vphantom{\overline{b}} a b c \right)} \cdot
    \overline{\left( a \overline{b} c \right)} \cdot
    \left( \overline{b} + c \right)
  }
\end{equation*}

Simplifiquem:

\begin{align*}
  f(a,b,c) &= \overline{
    \overline{\left(\vphantom{\overline{b}} a b c \right)} \cdot
    \overline{\left( a \overline{b} c \right)} \cdot
    \left( \overline{b} + c \right)
  }
\\
  &=
    \left( a b c \right) +
    \left( a \overline{b} c \right) +
    \overline{\left( \overline{b} + c \right)}
\\
  &=
    a b c +
    a \overline{b} c +
    b \overline{c}
\\
  &=
    a \left( b + \overline{b} \right) c +
    b \overline{c}
\\
  &=
    a c +
    b \overline{c}
\end{align*}

Fent el logigrama, resulta:

\begin{center} \begin{circuitikz}[scale=1] \draw
  (0,2) coordinate (a) node[anchor=east] {$a$}
  (0,1) coordinate (b) node[anchor=east] {$b$}
  (0,0) coordinate (c) node[anchor=east] {$c$}

  (0.7,0) node[not gate US, draw] (c_not) {}
  (2,1.5) node[and gate US, draw, logic gate inputs=nn] (and1) {}
  (2,0.5) node[and gate US, draw, logic gate inputs=nn] (and2) {}
  (3.2,1) node[or gate US, draw, logic gate inputs=nn] (or) {}

  (c_not.input) -- (c)
  (c) to[open, -*] ++(0.3,0) coordinate (c_yes)

  (and1.input 1) -- ++(-.2,0) |- (a)
  (and2.input 1) -- ++(-.2,0) |- (b)
  (and1.input 2) -- ++(-.2,0) -| (c_yes)
  (and2.input 2) -- ++(-.2,0) |- (c_not.output)

  (or.input 1) -- ++(-.2,0) |- (and1.output)
  (or.input 2) -- ++(-.2,0) |- (and2.output)

  (or.output) -- ++(.5,0) node[anchor=west] {$f$}
; \end{circuitikz} \end{center}

Per comprovar que el comportament és equivalent a l'enunciat,
fem la taula de veritat i veiem que coincideix.

\begin{center} \begin{tabular}{cc}
$cba$ & $f$ \\
\hline
000 & 0 \\
001 & 0 \\
010 & 1 \\
011 & 1 \\
100 & 0 \\
101 & 1 \\
110 & 0 \\
111 & 1
\end{tabular} \end{center}


\end{minipage}
\end{document}
