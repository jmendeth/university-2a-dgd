\documentclass[catalan,border=15pt,class=scrartcl,multi=minipage]{standalone}

% encoding
\usepackage[utf8]{inputenc}
\usepackage[T1]{fontenc}
\usepackage{lmodern}
\usepackage{babel}

% formatting and fixes
\frenchspacing
\usepackage[style=spanish]{csquotes}
\MakeAutoQuote{«}{»}

% ADD ANY SPECIFIC PACKAGES HERE
% (CHEMISTRY, CODE, PUBLISHING)
\usepackage{mathtools}
\usepackage{circuitikz}
\usetikzlibrary{circuits.logic.US,circuits.logic.IEC}
\usetikzlibrary{calc}

% other options
\setcounter{tocdepth}{6}

% hyperlink setup / metadata
\usepackage{hyperref}
\AfterPreamble{\hypersetup{
  pdfauthor={Xavier Mendez},
  pdftitle={T6. Problemes tema 2 (II)},
  pdfsubject={DGD},
}}

\begin{document}
\setlength{\parskip}{0pt}

\begin{minipage}{30em}

\paragraph{Problema 2.7} \hspace{0em}

\vspace{1em}


\subparagraph{Apartat A}

Escribim la taula de veritat de $Y_0$:

\begin{center} \begin{tabular}{cc}
$x_3x_2x_1x_0$ & $Y_0$ \\
\hline
0000 & 1 \\
0001 & 1 \\
0010 & 1 \\
0011 & 1 \\
0100 & 0 \\
0101 & 0 \\
0110 & 1 \\
0111 & 1 \\
1000 & 0 \\
1001 & 1 \\
1010 & 1 \\
1011 & 1 \\
1100 & 0 \\
1101 & 1 \\
1110 & 0 \\
1111 & 1
\end{tabular} \end{center}

La forma canònica en PdS es troba fent producte dels 5 maxterms que apareixen a la taula de veritat:
%
\begin{align*}
  Y_0(x_3,x_2,x_1,x_0) =&
    \left( x_3 + \overline{x_2} + x_1 + x_0 \right) \\ \cdot&
    \left( x_3 + \overline{x_2} + x_1 + \overline{x_0} \right) \\ \cdot&
    \left( \overline{x_3} + x_2 + x_1 + x_0 \right) \\ \cdot&
    \left( \overline{x_3} + \overline{x_2} + x_1 + x_0 \right) \\ \cdot&
    \left( \overline{x_3} + \overline{x_2} + \overline{x_1} + x_0 \right)
\end{align*}


\subparagraph{Apartat B}

Construïm el mapa de Karnaugh:

\begin{center} \begin{tabular}{ccccc}
\hspace{.2em} \tikz[baseline=.5em]{\node at (.8em,.8em) {$x_1x_0$}; \node at (-.8em,-.8em) {$x_3x_2$}; \draw (1em,-1em) -- (-1em,1em);}
   & 00 & 01 & 11 & 10 \\
00 &  1 &  1 &  1 &  1 \\
01 &  0 &  0 &  1 &  1 \\
11 &  0 &  1 &  1 &  0 \\
10 &  0 &  1 &  1 &  1
\end{tabular} \end{center}

Escribim l'expressió mínima en PdS:
%
\begin{equation*}
  Y_0(x_3,x_2,x_1,x_0) =
    \left( x_3 + \overline{x_2} + x_1 \right)
    \left( \overline{x_3} + \overline{x_2} + x_0 \right)
    \left( \overline{x_3} + x_1 + x_0 \right)
\end{equation*}

Apliquem Morgan al producte per obtenir la expressió amb \textsf{NOR} de \textsf{NOR}s:
%
\begin{equation*}
  Y_0(x_3,x_2,x_1,x_0) = \overline{
    \overline{x_3 + \overline{x_2} + x_1} +
    \overline{\overline{x_3} + \overline{x_2} + x_0} +
    \overline{\overline{x_3} + x_1 + x_0}
  }
\end{equation*}

El logigrama quedaria així:

\begin{center} \begin{circuitikz}[scale=1] \draw

(0.0,0) coordinate (d) node[anchor=south] {$x_3$} (d) -- ++(0,-5)
(d) ++(.5, -1) node[not gate US, draw, rotate=-90] (d_not) {}
(d_not.input) -- ++(0,.3) to[short, -*] ++(-.5,0)
(d) ++(.5,-5) -- (d_not.output)

(1.2,0) coordinate (d) node[anchor=south] {$x_2$} (d) -- ++(0,-5)
(d) ++(.5, -1) node[not gate US, draw, rotate=-90] (d_not) {}
(d_not.input) -- ++(0,.3) to[short, -*] ++(-.5,0)
(d) ++(.5,-5) -- (d_not.output)

(2.4,0) coordinate (d) node[anchor=south] {$x_1$} (d) -- ++(0,-5)
(d) ++(.5, -1) node[not gate US, draw, rotate=-90] (d_not) {}
(d_not.input) -- ++(0,.3) to[short, -*] ++(-.5,0)
(d) ++(.5,-5) -- (d_not.output)

(3.6,0) coordinate (d) node[anchor=south] {$x_0$} (d) -- ++(0,-5)
(d) ++(.5, -1) node[not gate US, draw, rotate=-90] (d_not) {}
(d_not.input) -- ++(0,.3) to[short, -*] ++(-.5,0)
(d) ++(.5,-5) -- (d_not.output)

;

\node at (5,-2) [nor gate US, draw, logic gate inputs=nnn] (and1) {};
\foreach \a in {1,...,3}
  \draw (and1.input \a -| -0.5,0) -- (and1.input \a);

\node at (5,-3) [nor gate US, draw, logic gate inputs=nnn] (and2) {};
\foreach \a in {1,...,3}
  \draw (and2.input \a -| -0.5,0) -- (and2.input \a);

\node at (5,-4) [nor gate US, draw, logic gate inputs=nnn] (and3) {};
\foreach \a in {1,...,3}
  \draw (and3.input \a -| -0.5,0) -- (and3.input \a);

\draw
  (6.4,-3) node[nor gate US, draw, logic gate inputs=nnn] (or) {}
  (or.input 1) -- ++(-0.2,0) |- (and1.output)
  (or.input 2) -- ++(-0.2,0) |- (and2.output)
  (or.input 3) -- ++(-0.2,0) |- (and3.output)
  (or.output) -- ++(0.5,0) node[anchor=west] {$Y_0$}
;

\draw
  (and1.input 1 -| 0.0,0)          node[circ] {}
  (and1.input 2 -| 1.2,0) ++(.5,0) node[circ] {}
  (and1.input 3 -| 2.4,0)          node[circ] {}

  (and2.input 1 -| 0.0,0) ++(.5,0) node[circ] {}
  (and2.input 2 -| 1.2,0) ++(.5,0) node[circ] {}
  (and2.input 3 -| 3.6,0)          node[circ] {}

  (and3.input 1 -| 0.0,0) ++(.5,0) node[circ] {}
  (and3.input 2 -| 2.4,0)          node[circ] {}
  (and3.input 3 -| 3.6,0)          node[circ] {}
;

\end{circuitikz} \end{center}


\subparagraph{Apartat C}

Escribim la taula de veritat de $Y_1$, junt amb la de $Y_0$ que hem escrit a dalt:

\begin{center} \begin{tabular}{ccc}
$x_3x_2x_1x_0$ & $Y_0$ & $Y_1$ \\
\hline
0000 & 1 & 1 \\
0001 & 1 & 0 \\
0010 & 1 & 1 \\
0011 & 1 & 0 \\
0100 & 0 & 0 \\
0101 & 0 & 0 \\
0110 & 1 & 0 \\
0111 & 1 & 1 \\
1000 & 0 & 1 \\
1001 & 1 & 1 \\
1010 & 1 & 1 \\
1011 & 1 & 1 \\
1100 & 0 & 0 \\
1101 & 1 & 1 \\
1110 & 0 & 0 \\
1111 & 1 & 1 \\
\end{tabular} \end{center}

D'acord amb això, les entrades $\left(0,0,0,1\right)$, $\left(0,0,1,1\right)$,
$\left(0,1,1,0\right)$ i $\left(1,0,0,0\right)$ han de ser inespecificacions ja
que tenen sortides diferents en les implementacions $Y_0$ i $Y_1$.


\subparagraph{Apartat D}

Fem el mapa de Karnaugh basant-nos en la taula de veritat de l'apartat B:

\begin{center} \begin{tabular}{ccccc}
\hspace{.2em} \tikz[baseline=.5em]{\node at (.8em,.8em) {$x_1x_0$}; \node at (-.8em,-.8em) {$x_3x_2$}; \draw (1em,-1em) -- (-1em,1em);}
   & 00 & 01 & 11 & 10 \\
00 &  1 &  0 &  0 &  1 \\
01 &  0 &  0 &  1 &  0 \\
11 &  0 &  1 &  1 &  0 \\
10 &  1 &  1 &  1 &  1
\end{tabular} \end{center}

Escribim una expressió simplificada en SdP:
%
\begin{equation*}
  Y_1(x_3,x_2,x_1,x_0) = x_3x_0 + x_2x_1x_0 + \overline{x_2}\overline{x_0}
\end{equation*}

Apliquem Morgan a la suma per convertir l'expressió en \textsf{NAND} de \textsf{NAND}s, i repetim factors en alguns productes per tal de tenir sempre 3 entrades:
%
\begin{equation*}
  Y_1(x_3,x_2,x_1,x_0) = \overline{x_3 \cdot x_0 \cdot x_0}
    + \overline{x_2 \cdot x_1 \cdot x_0}
    + \overline{\overline{x_2} \cdot \overline{x_0} \cdot \overline{x_0}}
\end{equation*}

El logigrama quedaria així:

\begin{center} \begin{circuitikz}[scale=1] \draw

(0.0,0) coordinate (d) node[anchor=south] {$x_3$} (d) -- ++(0,-5)
(d) ++(.5, -1) node[not gate US, draw, rotate=-90] (d_not) {}
(d_not.input) -- ++(0,.3) to[short, -*] ++(-.5,0)
(d) ++(.5,-5) -- (d_not.output)

(1.2,0) coordinate (d) node[anchor=south] {$x_2$} (d) -- ++(0,-5)
(d) ++(.5, -1) node[not gate US, draw, rotate=-90] (d_not) {}
(d_not.input) -- ++(0,.3) to[short, -*] ++(-.5,0)
(d) ++(.5,-5) -- (d_not.output)

(2.4,0) coordinate (d) node[anchor=south] {$x_1$} (d) -- ++(0,-5)
(d) ++(.5, -1) node[not gate US, draw, rotate=-90] (d_not) {}
(d_not.input) -- ++(0,.3) to[short, -*] ++(-.5,0)
(d) ++(.5,-5) -- (d_not.output)

(3.6,0) coordinate (d) node[anchor=south] {$x_0$} (d) -- ++(0,-5)
(d) ++(.5, -1) node[not gate US, draw, rotate=-90] (d_not) {}
(d_not.input) -- ++(0,.3) to[short, -*] ++(-.5,0)
(d) ++(.5,-5) -- (d_not.output)

;

\node at (5,-2) [nand gate US, draw, logic gate inputs=nnn] (and1) {};
\foreach \a in {1,...,3}
  \draw (and1.input \a -| -0.5,0) -- (and1.input \a);

\node at (5,-3) [nand gate US, draw, logic gate inputs=nnn] (and2) {};
\foreach \a in {1,...,3}
  \draw (and2.input \a -| -0.5,0) -- (and2.input \a);

\node at (5,-4) [nand gate US, draw, logic gate inputs=nnn] (and3) {};
\foreach \a in {1,...,3}
  \draw (and3.input \a -| -0.5,0) -- (and3.input \a);

\draw
  (6.4,-3) node[nand gate US, draw, logic gate inputs=nnn] (or) {}
  (or.input 1) -- ++(-0.2,0) |- (and1.output)
  (or.input 2) -- ++(-0.2,0) |- (and2.output)
  (or.input 3) -- ++(-0.2,0) |- (and3.output)
  (or.output) -- ++(0.5,0) node[anchor=west] {$Y_1$}
;

\draw
  (and1.input 1 -| 0.0,0)          node[circ] {}
  (and1.input 2 -| 3.6,0)          node[circ] {}
  (and1.input 3 -| 3.6,0)          node[circ] {}

  (and2.input 1 -| 1.2,0)          node[circ] {}
  (and2.input 2 -| 2.4,0)          node[circ] {}
  (and2.input 3 -| 3.6,0)          node[circ] {}

  (and3.input 1 -| 1.2,0) ++(.5,0) node[circ] {}
  (and3.input 2 -| 3.6,0) ++(.5,0) node[circ] {}
  (and3.input 3 -| 3.6,0) ++(.5,0) node[circ] {}
;

\end{circuitikz} \end{center}


\end{minipage}

\end{document}
