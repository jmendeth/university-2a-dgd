\documentclass[catalan,border=15pt,class=scrartcl,multi=minipage,parskip=half*]{standalone}

% encoding
\usepackage[utf8]{inputenc}
\usepackage[T1]{fontenc}
\usepackage{lmodern}
\usepackage{babel}

% formatting and fixes
\frenchspacing
\usepackage[style=spanish]{csquotes}
\MakeAutoQuote{«}{»}
\usepackage{bookmark}

% ADD ANY SPECIFIC PACKAGES HERE
% (CHEMISTRY, CODE, PUBLISHING)
\usepackage{mathtools}
\usepackage{circuitikz}
\usetikzlibrary{calc}
\usetikzlibrary{automata}
\usetikzlibrary{circuits.logic.US,circuits.logic.IEC}
\usetikzlibrary{circuits.logic.mux}
%\usepackage{karnaugh}

% other options
\setcounter{tocdepth}{6}
\setcounter{secnumdepth}{2}

% hyperlink setup / metadata
\usepackage{hyperref}
\AfterPreamble{\hypersetup{
  pdfauthor={Xavier Mendez},
  pdfsubject={DGD},
}}

% custom commands
\newcommand{\startpage}{\begin{minipage}{30em}}
\newcommand{\finishpage}{\end{minipage}}
\newcommand{\iopair}[2]{\( \left(#1\right) \rightarrow #2 \)}

%FIXME: don't use circ/to short, it's from circuitikz

\AfterPreamble{\hypersetup{
  pdftitle={T6. Problemes tema 2 (II)},
}}

\begin{document}

\startpage
\paragraph{Problema 2.7} \hspace{0em}

\vspace{1em}


\subparagraph{Apartat A}

Escrivim la taula de veritat de $Y_0$:

\begin{center} \begin{tabular}{cc}
$x_3x_2x_1x_0$ & $Y_0$ \\
\hline
0000 & 1 \\
0001 & 1 \\
0010 & 1 \\
0011 & 1 \\
0100 & 0 \\
0101 & 0 \\
0110 & 1 \\
0111 & 1 \\
1000 & 0 \\
1001 & 1 \\
1010 & 1 \\
1011 & 1 \\
1100 & 0 \\
1101 & 1 \\
1110 & 0 \\
1111 & 1
\end{tabular} \end{center}

La forma canònica en PdS es troba fent producte dels 5 maxterms que apareixen a la taula de veritat:
%
\begin{align*}
  Y_0(x_3,x_2,x_1,x_0) =&
    \left( x_3 + \overline{x_2} + x_1 + x_0 \right) \\ \cdot&
    \left( x_3 + \overline{x_2} + x_1 + \overline{x_0} \right) \\ \cdot&
    \left( \overline{x_3} + x_2 + x_1 + x_0 \right) \\ \cdot&
    \left( \overline{x_3} + \overline{x_2} + x_1 + x_0 \right) \\ \cdot&
    \left( \overline{x_3} + \overline{x_2} + \overline{x_1} + x_0 \right)
\end{align*}


\subparagraph{Apartat B}

Construïm el mapa de Karnaugh:

\begin{center} \begin{tabular}{ccccc}
\hspace{.2em} \tikz[baseline=.5em]{\node at (.8em,.8em) {$x_1x_0$}; \node at (-.8em,-.8em) {$x_3x_2$}; \draw (1em,-1em) -- (-1em,1em);}
   & 00 & 01 & 11 & 10 \\
00 &  1 &  1 &  1 &  1 \\
01 &  0 &  0 &  1 &  1 \\
11 &  0 &  1 &  1 &  0 \\
10 &  0 &  1 &  1 &  1
\end{tabular} \end{center}

Escrivim l'expressió mínima en PdS:
%
\begin{equation*}
  Y_0(x_3,x_2,x_1,x_0) =
    \left( x_3 + \overline{x_2} + x_1 \right)
    \left( \overline{x_3} + \overline{x_2} + x_0 \right)
    \left( \overline{x_3} + x_1 + x_0 \right)
\end{equation*}

Apliquem Morgan al producte per obtenir l'expressió en \textsf{NOR} de \textsf{NOR}s:
%
\begin{equation*}
  Y_0(x_3,x_2,x_1,x_0) = \overline{
    \overline{x_3 + \overline{x_2} + x_1} +
    \overline{\overline{x_3} + \overline{x_2} + x_0} +
    \overline{\overline{x_3} + x_1 + x_0}
  }
\end{equation*}

El logigrama quedaria així:

\begin{center} \begin{tikzpicture}[circuit logic US] \draw

(0.0,0) coordinate (d) node[anchor=south] {$x_3$} (d) -- ++(0,-5)
(d) ++(.5, -1) node[not gate US, draw, point down] (d_not) {}
(d_not.input) -- ++(0,.3) to[short, -*] ++(-.5,0)
(d) ++(.5,-5) -- (d_not.output)

(1.2,0) coordinate (d) node[anchor=south] {$x_2$} (d) -- ++(0,-5)
(d) ++(.5, -1) node[not gate US, draw, point down] (d_not) {}
(d_not.input) -- ++(0,.3) to[short, -*] ++(-.5,0)
(d) ++(.5,-5) -- (d_not.output)

(2.4,0) coordinate (d) node[anchor=south] {$x_1$} (d) -- ++(0,-5)
(d) ++(.5, -1) node[not gate US, draw, point down] (d_not) {}
(d_not.input) -- ++(0,.3) to[short, -*] ++(-.5,0)
(d) ++(.5,-5) -- (d_not.output)

(3.6,0) coordinate (d) node[anchor=south] {$x_0$} (d) -- ++(0,-5)
(d) ++(.5, -1) node[not gate US, draw, point down] (d_not) {}
(d_not.input) -- ++(0,.3) to[short, -*] ++(-.5,0)
(d) ++(.5,-5) -- (d_not.output)

;

\node at (5,-2) [nor gate, draw, inputs=nnn] (and1) {};
\foreach \a in {1,...,3}
  \draw (and1.input \a -| -0.5,0) -- (and1.input \a);

\node at (5,-3) [nor gate, draw, inputs=nnn] (and2) {};
\foreach \a in {1,...,3}
  \draw (and2.input \a -| -0.5,0) -- (and2.input \a);

\node at (5,-4) [nor gate, draw, inputs=nnn] (and3) {};
\foreach \a in {1,...,3}
  \draw (and3.input \a -| -0.5,0) -- (and3.input \a);

\draw
  (6.4,-3) node[nor gate, draw, inputs=nnn] (or) {}
  (or.input 1) -- ++(-0.2,0) |- (and1.output)
  (or.input 2) -- ++(-0.2,0) |- (and2.output)
  (or.input 3) -- ++(-0.2,0) |- (and3.output)
  (or.output) -- ++(0.5,0) node[anchor=west] {$Y_0$}
;

\draw
  (and1.input 1 -| 0.0,0)          node[circ] {}
  (and1.input 2 -| 1.2,0) ++(.5,0) node[circ] {}
  (and1.input 3 -| 2.4,0)          node[circ] {}

  (and2.input 1 -| 0.0,0) ++(.5,0) node[circ] {}
  (and2.input 2 -| 1.2,0) ++(.5,0) node[circ] {}
  (and2.input 3 -| 3.6,0)          node[circ] {}

  (and3.input 1 -| 0.0,0) ++(.5,0) node[circ] {}
  (and3.input 2 -| 2.4,0)          node[circ] {}
  (and3.input 3 -| 3.6,0)          node[circ] {}
;

\end{tikzpicture} \end{center}


\subparagraph{Apartat C}

Escrivim la taula de veritat de $Y_1$, junt amb la de $Y_0$ que hem escrit a dalt:

\begin{center} \begin{tabular}{ccc}
$x_3x_2x_1x_0$ & $Y_0$ & $Y_1$ \\
\hline
0000 & 1 & 1 \\
0001 & 1 & 0 \\
0010 & 1 & 1 \\
0011 & 1 & 0 \\
0100 & 0 & 0 \\
0101 & 0 & 0 \\
0110 & 1 & 0 \\
0111 & 1 & 1 \\
1000 & 0 & 1 \\
1001 & 1 & 1 \\
1010 & 1 & 1 \\
1011 & 1 & 1 \\
1100 & 0 & 0 \\
1101 & 1 & 1 \\
1110 & 0 & 0 \\
1111 & 1 & 1 \\
\end{tabular} \end{center}

D'acord amb això, les entrades $\left(0,0,0,1\right)$, $\left(0,0,1,1\right)$,
$\left(0,1,1,0\right)$ i $\left(1,0,0,0\right)$ han de ser inespecificacions ja
que tenen sortides diferents en les implementacions $Y_0$ i $Y_1$.


\subparagraph{Apartat D}

Fem el mapa de Karnaugh basant-nos en la taula de veritat de l'apartat B:

\begin{center} \begin{tabular}{ccccc}
\hspace{.2em} \tikz[baseline=.5em]{\node at (.8em,.8em) {$x_1x_0$}; \node at (-.8em,-.8em) {$x_3x_2$}; \draw (1em,-1em) -- (-1em,1em);}
   & 00 & 01 & 11 & 10 \\
00 &  1 &  0 &  0 &  1 \\
01 &  0 &  0 &  1 &  0 \\
11 &  0 &  1 &  1 &  0 \\
10 &  1 &  1 &  1 &  1
\end{tabular} \end{center}

Escrivim una expressió simplificada en SdP:
%
\begin{equation*}
  Y_1(x_3,x_2,x_1,x_0) = x_3x_0 + x_2x_1x_0 + \overline{x_2}\overline{x_0}
\end{equation*}

Apliquem Morgan a la suma per obtenir l'expressió en \textsf{NAND} de \textsf{NAND}s, i repetim factors en alguns productes per tal de tenir sempre 3 entrades:
%
\begin{equation*}
  Y_1(x_3,x_2,x_1,x_0) = \overline{
    \overline{x_3 \cdot x_0 \cdot x_0} \cdot
    \overline{x_2 \cdot x_1 \cdot x_0} \cdot
    \overline{\overline{x_2} \cdot \overline{x_0} \cdot \overline{x_0}}
  }
\end{equation*}

El logigrama quedaria així:

\begin{center} \begin{tikzpicture}[circuit logic US] \draw

(0.0,0) coordinate (d) node[anchor=south] {$x_3$} (d) -- ++(0,-5)
(d) ++(.5, -1) node[not gate US, draw, point down] (d_not) {}
(d_not.input) -- ++(0,.3) to[short, -*] ++(-.5,0)
(d) ++(.5,-5) -- (d_not.output)

(1.2,0) coordinate (d) node[anchor=south] {$x_2$} (d) -- ++(0,-5)
(d) ++(.5, -1) node[not gate US, draw, point down] (d_not) {}
(d_not.input) -- ++(0,.3) to[short, -*] ++(-.5,0)
(d) ++(.5,-5) -- (d_not.output)

(2.4,0) coordinate (d) node[anchor=south] {$x_1$} (d) -- ++(0,-5)
(d) ++(.5, -1) node[not gate US, draw, point down] (d_not) {}
(d_not.input) -- ++(0,.3) to[short, -*] ++(-.5,0)
(d) ++(.5,-5) -- (d_not.output)

(3.6,0) coordinate (d) node[anchor=south] {$x_0$} (d) -- ++(0,-5)
(d) ++(.5, -1) node[not gate US, draw, point down] (d_not) {}
(d_not.input) -- ++(0,.3) to[short, -*] ++(-.5,0)
(d) ++(.5,-5) -- (d_not.output)

;

\node at (5,-2) [nand gate, draw, inputs=nnn] (and1) {};
\foreach \a in {1,...,3}
  \draw (and1.input \a -| -0.5,0) -- (and1.input \a);

\node at (5,-3) [nand gate, draw, inputs=nnn] (and2) {};
\foreach \a in {1,...,3}
  \draw (and2.input \a -| -0.5,0) -- (and2.input \a);

\node at (5,-4) [nand gate, draw, inputs=nnn] (and3) {};
\foreach \a in {1,...,3}
  \draw (and3.input \a -| -0.5,0) -- (and3.input \a);

\draw
  (6.4,-3) node[nand gate, draw, inputs=nnn] (or) {}
  (or.input 1) -- ++(-0.2,0) |- (and1.output)
  (or.input 2) -- ++(-0.2,0) |- (and2.output)
  (or.input 3) -- ++(-0.2,0) |- (and3.output)
  (or.output) -- ++(0.5,0) node[anchor=west] {$Y_1$}
;

\draw
  (and1.input 1 -| 0.0,0)          node[circ] {}
  (and1.input 2 -| 3.6,0)          node[circ] {}
  (and1.input 3 -| 3.6,0)          node[circ] {}

  (and2.input 1 -| 1.2,0)          node[circ] {}
  (and2.input 2 -| 2.4,0)          node[circ] {}
  (and2.input 3 -| 3.6,0)          node[circ] {}

  (and3.input 1 -| 1.2,0) ++(.5,0) node[circ] {}
  (and3.input 2 -| 3.6,0) ++(.5,0) node[circ] {}
  (and3.input 3 -| 3.6,0) ++(.5,0) node[circ] {}
;

\end{tikzpicture} \end{center}

\finishpage


\startpage
\paragraph{Problema 2.8} \hspace{0em}

\subparagraph{Apartat A}

Mentalment transformem el logigrama a \textsf{AND} de \textsf{OR}s i escrivim les sortides del primer nivell de portes, que enumerarem de dalt a baix:
%
\begin{align*}
  t_0 &= s + \overline{x_1} \\
  t_1 &= x_1 + x_0 \\
  t_2 &= s + x_1 \\
  t_3 &= \overline{s} + \overline{x_1} + \overline{x_0} \\
  t_4 &= \overline{x_1} + x_0
\end{align*}

Escrivim les expressions de $z_2$, $z_1$ i $z_0$ en funció de les sortides de dalt:
%
\begin{align*}
  z_2 &= t_0 t_1 t_2 \\
  z_1 &= t_1 t_2 t_3 \\
  z_0 &= t_1 t_4
\end{align*}

Escrivim la taula de veritat de $z_2$, $z_1$ i $z_0$:

\begin{center} \begin{tabular}{cccc}
$sx_1x_0$ & $z_2$ & $z_1$ & $z_0$ \\
\hline
000 & 0 & 0 & 0 \\
001 & 0 & 0 & 1 \\
010 & 0 & 1 & 0 \\
011 & 0 & 1 & 1 \\
100 & 0 & 0 & 0 \\
101 & 1 & 1 & 1 \\
110 & 1 & 1 & 0 \\
111 & 1 & 0 & 1
\end{tabular} \end{center}

Escrivim els mapes de Karnaugh per a $z_2$...

\begin{center} \begin{tabular}{lcccc}
\hspace{-.7em} \tikz[baseline=.1em]{\node at (.5em,.5em) {$x_1x_0$}; \node at (-.5em,-.5em) {$s$}; \draw (1em,-1em) -- (-1em,1em);}
  & 00 & 01 & 11 & 10 \\
0 &  0 &  0 &  0 &  0 \\
1 &  0 &  1 &  1 &  1 \\
\end{tabular} \end{center}

Per a $z_1$...

\begin{center} \begin{tabular}{lcccc}
\hspace{-.7em} \tikz[baseline=.1em]{\node at (.5em,.5em) {$x_1x_0$}; \node at (-.5em,-.5em) {$s$}; \draw (1em,-1em) -- (-1em,1em);}
  & 00 & 01 & 11 & 10 \\
0 &  0 &  0 &  1 &  1 \\
1 &  0 &  1 &  0 &  1 \\
\end{tabular} \end{center}

Per a $z_0$...

\begin{center} \begin{tabular}{lcccc}
\hspace{-.7em} \tikz[baseline=.1em]{\node at (.5em,.5em) {$x_1x_0$}; \node at (-.5em,-.5em) {$s$}; \draw (1em,-1em) -- (-1em,1em);}
  & 00 & 01 & 11 & 10 \\
0 &  0 &  1 &  1 &  0 \\
1 &  0 &  1 &  1 &  0 \\
\end{tabular} \end{center}

Escrivim les expressions simplificades en PdS:
%
\begin{align*}
  z_2 &= \left(x_1 + x_0\right) s \\
  z_1 &= \left(x_1 + x_0\right) \left(s + x_1\right) \left(\overline{s} + \overline{x_1} + \overline{x_0}\right) \\
  z_0 &= x_0
\end{align*}


\subparagraph{Apartat B}

Escrivim les expressions simplificades ara en SdP:
%
\begin{align*}
  z_2 &= s \overline{x_1} x_0 + s x_1 \\
  z_1 &= s \overline{x_1} x_0 + \overline{s} x_1 + x_1 \overline{x_0} \\
  z_0 &= x_0
\end{align*}

Fixem-nos que, tot i que existeixen expressions més simples individualment per a $z_2$, la que hem fet servir ens beneficia perquè reutilitza productes de $z_1$, reduïnt el nombre de portes lògiques necessàries.

Ara, apliquem Morgan a les sumes per obtenir les expressions en \textsf{NAND} de \textsf{NAND}s (fent simplificacions trivials tipus $\overline{\overline{x}} = x$):
%
\begin{align*}
  z_2 &= \overline{
    \overline{\left(s \overline{x_1} x_0\right)} \cdot
    \overline{\left(s x_1\right)}
  } \\
  z_1 &= \overline{
    \overline{\left(s \overline{x_1} x_0\right)} \cdot
    \overline{\left(\overline{s} x_1\right)} \cdot
    \overline{\left(x_1 \overline{x_0}\right)}
  } \\
  z_0 &= x_0
\end{align*}

Ara, comptem que hi ha 4 productes negats únics en el primer nivell, i 2 en el nivell superior. El nombre de portes \textsf{NAND} mínimes per implementar-les, sense comptar la implementació dels negats, seria \textbf{6}.


\subparagraph{Apartat C}

El circuit serveix per a canviar del codi d'entrada (signe -- valor absolut) a Ca2 de 3~bits.

\finishpage


\startpage
\paragraph{Problema 2.16}

Suposant que l'entrada de la funció ha sigut $(A,B)$ durant un temps suficientment llarg, es diu que hi ha hagut un espuri estàtic si, en aplicar una nova entrada $(A',B')$, s'observen \emph{dos} canvis de sortida abans d'arribar a la sortida definitiva. Aplicant la definició en la nostra xarxa, hi ha hagut espuri estàtic si i només si:
%
\begin{equation*}
  \overline{AB + \overline{A+B}} \neq \overline{A'B' + \overline{A+B}}
    \quad \text{i} \quad
  \overline{A'B' + \overline{A+B}} \neq \overline{A'B' + \overline{A'+B'}}
\end{equation*}

Manipulem:
%
\begin{align*}
  AB + \overline{A+B} \neq A'B' + \overline{A+B}
    \quad &\text{i} \quad
  A'B' + \overline{A+B} \neq A'B' + \overline{A'+B'}
\\
  AB \neq A'B'
    \quad \text{i} \quad
  \overline{A+B} = 0
    \quad &\text{i} \quad
  A'B' = 0
    \quad \text{i} \quad
  \overline{A+B} \neq \overline{A'+B'}
\\
  AB \neq 0
    \quad \text{i} \quad
  A+B = 1
    \quad &\text{i} \quad
  A'B' = 0
    \quad \text{i} \quad
  0 \neq \overline{A'+B'}
\\
  AB = 1
    \quad \text{i} \quad
  A+B = 1
    \quad &\text{i} \quad
  A'B' = 0
    \quad \text{i} \quad
  0 = A'+B'
\\
  A = 1
    \quad \text{i} \quad
  B = 1
    \quad &\text{i} \quad
  A' = 0
    \quad \text{i} \quad
  B' = 0
\end{align*}

Per tant, només es produïrà un espuri estàtic quan l'entrada canviï de $\left(1,1\right)$ a $\left(0,0\right)$. La resposta a l'apartat~A és \emph{no}; la resposta a l'apartat~B és \emph{sí}.

\finishpage


\startpage
\paragraph{Problema 2.18} Visualment es prepara el següent anàlisi modular:

\vspace{1.3em} % hack for proper alignment
\begin{center} \begin{tikzpicture}[circuit logic US]
  \node[mux, inputs=nn] at (1,1.5) (mux) {};
  \draw
    (0,0) node[or gate, inputs=nnn] (m_or) {}
    (m_or.input 1) -- ++(-.2,0) |- ++(-.3,.2) node[anchor=east] {$m_0$}
    (m_or.input 2) -- ++(-.5,0) node[anchor=east] {$m_1$}
    (m_or.input 3) -- ++(-.2,0) |- ++(-.3,-.2) node[anchor=east] {$m_2$}
  ;
  \draw (m_or.output) -| (mux.select);
  \draw[ultra thick, <-] (mux.input 1) -- ++(-.5,0) node[anchor=east] {$X$};
  \draw[ultra thick, <-] (mux.input 2) -- ++(-.5,0) node[anchor=east] {$Y$};

  \node[rectangle, minimum height=1cm, minimum width=1cm, draw] at (2.8,1.5) (xb) {\textsf{TM}};
  \draw[ultra thick, ->] (mux.output) -- (xb);
  \node[anchor=south] at ($(mux.output)!.5!(xb)$) {$T$};
  \draw[ultra thick] (xb) -- ++(1,0) node[anchor=west] {$Z$};
\end{tikzpicture} \end{center}

Ara només queda analitzar el bloc \textsf{TM} i hauriem de ser capaços de donar una especificació en alt nivell de la funció del circuit. Escrivim la taula de veritat de \textsf{TM}:

\begin{center} \begin{tabular}{cc}
$t_2t_1t_0$ & $z_2z_1z_0$ \\
\hline
000 & 001 \\
001 & 010 \\
010 & 011 \\
011 & 100 \\
100 & 101 \\
101 & 110 \\
110 & 111 \\
111 & 100
\end{tabular} \end{center}

Segons aquesta taula de veritat, sembla que \textsf{TM} suma $1$ a l'entrada, en binari natural, però quan l'entrada és $7$ retorna $4$ i no $0$ (possible inespecificació?). Una especificació en alt nivell seria:

\begin{displayquote}
La xarxa suma 1 a $Y$ i retorna el resultat. Excepcionalment, si $Y=7$ es retorna 4.

En cas que $M=0$, es fa servir $X$ en comptes d'$Y$.
\end{displayquote}

\finishpage


\startpage
\paragraph{Problema 2.19} \hspace{0em}

\subparagraph{Apartat A}

La xarxa retorna la distància entre $A$ i $B$: $C = \left|A - B\right|$

\subparagraph{Apartat C} El disseny per a \textsf{M\textsubscript{2}} és un multiplexor de 2~busos de 4~bits connectat apropiadament:

\vspace{1.5em} % hack for proper alignment
\begin{center} \begin{tikzpicture}[circuit logic US]
  \node[mux, inputs=nn] at (1,1.5) (mux) {};
  \draw (mux.select) -- ++(0,-.5) node[anchor=north] {$S$};
  \draw[ultra thick, <-] (mux.input 1) -- ++(-.5,0) node[anchor=east] {$V$};
  \draw[ultra thick, <-] (mux.input 2) -- ++(-.5,0) node[anchor=east] {$W$};
  \draw[ultra thick] (mux.output) -- ++(.5,0) node[anchor=west] {$U$};
\end{tikzpicture} \end{center}

\subparagraph{Apartat D} Sabem que \emph{per a aquest us en concret} de \textsf{M\textsubscript{3}}, $X \ge Y$, i per tant $0 \ge Z \ge 2^4-1$.

Es pot comprovar que un \emph{restador en Ca2} es comporta com un \emph{restador modular en binari natural}. I com que, per a les nostres entrades, la resta modular es comporta de la mateixa forma que la resta típica, podem fer servir un restador en Ca2 de 4 bits, sense preocupar-nos de fer cap conversió.

Un restador en Ca2 es pot fer amb un negador i un sumador amb carry d'entrada 1, connectats adequadament:

\begin{center} \begin{tikzpicture}[circuit logic US]
  \node[rectangle, draw, minimum height=2cm, minimum width=2cm] at (0,0) (sum) {$\sum$};
  \draw[font=\scriptsize, inner sep=2pt]
    (sum) ++(-.5,1) node[anchor=north] {$A$}
    (sum) ++(+.5,1) node[anchor=north] {$B$}
    (sum) ++(0,-1) node[anchor=south] {$Z$}
    (sum) ++(+1,0) node[anchor=east] {$C_i$}
    (sum) ++(-1,0) node[anchor=west] {$C_o$}
  ;

  \node[not gate US, draw, point down] at (.5,1.8) (not) {};

  \draw[ultra thick, <-] (sum) ++(-.5,1) -- (-.5,2.4) node[anchor=south] {$A$};
  \draw[ultra thick] (not.input) -- (.5,2.4) node[anchor=south] {$B$};
  \draw[ultra thick, <-] (sum) ++(.5,1) -- (not.output);
  \draw[ultra thick] (sum) ++(0,-1) -- ++(0,-.5) node[anchor=north] {$C$};

  \draw (sum) ++(1,0) -- ++(.4,0) node[anchor=west] {$1$};
\end{tikzpicture} \end{center}

\finishpage

\end{document}
