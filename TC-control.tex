\documentclass[catalan,border=15pt,class=scrartcl,multi=minipage,parskip=half*]{standalone}

% encoding
\usepackage[utf8]{inputenc}
\usepackage[T1]{fontenc}
\usepackage{lmodern}
\usepackage{babel}

% formatting and fixes
\frenchspacing
\usepackage[style=spanish]{csquotes}
\MakeAutoQuote{«}{»}
\usepackage{bookmark}

% ADD ANY SPECIFIC PACKAGES HERE
% (CHEMISTRY, CODE, PUBLISHING)
\usepackage{mathtools}
\usepackage{circuitikz}
\usetikzlibrary{calc}
\usetikzlibrary{automata}
\usetikzlibrary{circuits.logic.US,circuits.logic.IEC}
\usetikzlibrary{circuits.logic.mux}
%\usepackage{karnaugh}

% other options
\setcounter{tocdepth}{6}
\setcounter{secnumdepth}{2}

% hyperlink setup / metadata
\usepackage{hyperref}
\AfterPreamble{\hypersetup{
  pdfauthor={Xavier Mendez},
  pdfsubject={DGD},
}}

% custom commands
\newcommand{\startpage}{\begin{minipage}{30em}}
\newcommand{\finishpage}{\end{minipage}}
\newcommand{\iopair}[2]{\( \left(#1\right) \rightarrow #2 \)}

%FIXME: don't use circ/to short, it's from circuitikz

\AfterPreamble{\hypersetup{
  pdftitle={Parcial 1},
}}

\begin{document}

\startpage
\paragraph{Problema 1}

\subparagraph{Apartat A}

El codi ha de ser autocomplementari, i si tots els pesos son positius la codificació 1111 ha de correspondre al valor $9$.
Per a què això sigui possible, $1 + 2 + 1 + x = 9 \Leftrightarrow x = 5$.

Per tant $x$, en cas d'existir i ser positiu, ha de ser $5$.

\subparagraph{Apartat B}

Assignem codificacions per als primers 5 valors:
%
\begin{align*}
  0000 &\rightarrow 0 \\
  0010 &\rightarrow 1 \\
  0100 &\rightarrow 2 \\
  0110 &\rightarrow 3 \\
  1110 &\rightarrow 4
\end{align*}
%
La resta de codificacions queden determinades perquè el codi és autocomplementari:
%
\begin{align*}
  0001 &\rightarrow 5 \\
  1001 &\rightarrow 6 \\
  1011 &\rightarrow 7 \\
  1101 &\rightarrow 8 \\
  1111 &\rightarrow 9
\end{align*}

Tot quadra, tenim un codi BCD ponderat autocomplementari vàlid.

\subparagraph{Apartat C}

Una opció és assignar pes $-9$ al cinquè bit. Llavors, quan estigui activat, les combinacions
dels bits existents recorren el rang $\left[-9,0\right]$. Només caldria descartar una de
les dues representacions de $0$ resultants. El codi podria quedar així:
%
\begin{align*}
  10000 &\rightarrow -9 \\
  10010 &\rightarrow -8 \\
  10100 &\rightarrow -7 \\
  10110 &\rightarrow -6 \\
  11110 &\rightarrow -5 \\
  10001 &\rightarrow -4 \\
  11001 &\rightarrow -3 \\
  11011 &\rightarrow -2 \\
  11101 &\rightarrow -1 \\
  00000 &\rightarrow  0 \\
  00010 &\rightarrow +1 \\
  00100 &\rightarrow +2 \\
  00110 &\rightarrow +3 \\
  01110 &\rightarrow +4 \\
  00001 &\rightarrow +5 \\
  01001 &\rightarrow +6 \\
  01011 &\rightarrow +7 \\
  01101 &\rightarrow +8 \\
  01111 &\rightarrow +9
\end{align*}

Que a més té la particularitat de que segueix sent autocomplementari excepte per
la doble representació del 0.

\finishpage

\startpage
\paragraph{Problema 2}

\subparagraph{Apartat A}

Escrivim la taula de veritat:

\begin{center} \begin{tabular}{ccccccc}
$b_3 b_2 b_1 b_0$ & $y_5$ & $y_4$ & $y_3$ & $y_2$ & $y_1$ & $y_0$ \\
\hline
0000 & 0 & 0 & 0 & 0 & 0 & 0 \\
0001 & 0 & 0 & 0 & 1 & 0 & 1 \\
0010 & 0 & 0 & 1 & 0 & 1 & 0 \\
0011 & 0 & 0 & 1 & 1 & 1 & 1 \\
0100 & 0 & 1 & 0 & 1 & 0 & 0 \\
0101 & 0 & 1 & 1 & 0 & 0 & 1 \\
0110 & 0 & 1 & 1 & 1 & 1 & 0 \\
0111 & 1 & 0 & 0 & 0 & 1 & 1 \\
1000 & 1 & 0 & 1 & 0 & 0 & 0 \\
1001 & 1 & 0 & 1 & 1 & 0 & 1 \\
1010 &-- &-- &-- &-- &-- &-- \\
1011 &-- &-- &-- &-- &-- &-- \\
1100 &-- &-- &-- &-- &-- &-- \\
1101 &-- &-- &-- &-- &-- &-- \\
1110 &-- &-- &-- &-- &-- &-- \\
1111 &-- &-- &-- &-- &-- &--
\end{tabular} \end{center}

Per a $y_0$ i $y_1$, es veu a simple vista que:
%
\begin{align*}
  y_0 &= b_0 \\
  y_1 &= b_1
\end{align*}

Per a les altres quatre sortides, fem el mapa de Karnaugh.

Per a $y_2$:

\begin{center} \begin{tabular}{lcccc}
\hspace{-.7em} \tikz[baseline=.1em]{\node at (.5em,.5em) {$b_1b_0$}; \node at (-.5em,-.5em) {$b_3b_2$}; \draw (1em,-1em) -- (-1em,1em);}
   & 00 & 01 & 11 & 10 \\
00 &  0 &  1 &  1 &  0 \\
01 &  1 &  0 &  0 &  1 \\
11 & -- & -- & -- & -- \\
10 &  0 &  1 & -- & --
\end{tabular} \end{center}
%
\begin{equation*}
  y_2 = b_2 \overline{b_0} + \overline{b_2} b_0
\end{equation*}

Per a $y_3$:

\begin{center} \begin{tabular}{lcccc}
\hspace{-.7em} \tikz[baseline=.1em]{\node at (.5em,.5em) {$b_1b_0$}; \node at (-.5em,-.5em) {$b_3b_2$}; \draw (1em,-1em) -- (-1em,1em);}
   & 00 & 01 & 11 & 10 \\
00 &  0 &  0 &  1 &  1 \\
01 &  0 &  1 &  0 &  1 \\
11 & -- & -- & -- & -- \\
10 &  1 &  1 & -- & --
\end{tabular} \end{center}
%
\begin{equation*}
  y_3 = b_2 \overline{b_1} b_0 + b_1 \overline{b_2} + b_1 \overline{b_0} + b_3
\end{equation*}

Per a $y_4$:

\begin{center} \begin{tabular}{lcccc}
\hspace{-.7em} \tikz[baseline=.1em]{\node at (.5em,.5em) {$b_1b_0$}; \node at (-.5em,-.5em) {$b_3b_2$}; \draw (1em,-1em) -- (-1em,1em);}
   & 00 & 01 & 11 & 10 \\
00 &  0 &  0 &  0 &  0 \\
01 &  1 &  1 &  0 &  1 \\
11 & -- & -- & -- & -- \\
10 &  0 &  0 & -- & --
\end{tabular} \end{center}
%
\begin{equation*}
  y_4 = b_2 \overline{b_0} + b_2 \overline{b_1}
\end{equation*}

Per a $y_5$:

\begin{center} \begin{tabular}{lcccc}
\hspace{-.7em} \tikz[baseline=.1em]{\node at (.5em,.5em) {$b_1b_0$}; \node at (-.5em,-.5em) {$b_3b_2$}; \draw (1em,-1em) -- (-1em,1em);}
   & 00 & 01 & 11 & 10 \\
00 &  0 &  0 &  0 &  0 \\
01 &  0 &  0 &  1 &  0 \\
11 & -- & -- & -- & -- \\
10 &  1 &  1 & -- & --
\end{tabular} \end{center}
%
\begin{equation*}
  y_5 = b_3 + b_2 b_1 b_0
\end{equation*}

\subparagraph{Apartat B}

La funció que estem realitzant té quatre entrades i sis sortides. Cap sortida és
suma de més de 4 productes.

Per tant, la PAL mínima per a realitzar el bloc \textsf{B1} consta de 4 entrades,
4 productes i 6 sortides.

\subparagraph{Apartat C}

A simple vista es veu que una combinació $a_3...a_0$ no pertany al codi BCD
si $a_3 \cdot \left( a_2 + a_1 \right) = 1$.

Per tant, el bloc \textsf{ERR} es podria descriure per l'expressió:
%
\begin{equation*}
 ERR = a_3 \cdot \left( a_2 + a_1 \right) + b_3 \cdot \left( b_2 + b_1 \right)
\end{equation*}

La taula de veritat (parcial) en funció de $a_3$ i $b_3$ és:

\begin{center} \begin{tabular}{ccccccc}
$b_3$ & $a_3$ & $ERR$ \\
\hline
0 & 0 & $0$ \\
0 & 1 & $a_2 + a_1$ \\
1 & 0 & $b_2 + b_1$ \\
1 & 1 & $\left( a_2 + a_1 \right) + \left( b_2 + b_1 \right)$
\end{tabular} \end{center}

El logigrama amb un MUX 4:1 quedaria així:

\begin{center} \begin{tikzpicture}[circuit logic US, circuit ee IEC]
  \node[mux, inputs=nnnn] at (3,0) (mux) {};
  \draw
    (0,.5) node[or gate, inputs=nn] (or_a) {}
    (0,-.5) node[or gate, inputs=nn] (or_b) {}
    (or_a.input 1) -- ++(-.5,0) node[anchor=east] {$a_2$}
    (or_a.input 2) -- ++(-.5,0) node[anchor=east] {$a_1$}
    (or_b.input 1) -- ++(-.5,0) node[anchor=east] {$b_2$}
    (or_b.input 2) -- ++(-.5,0) node[anchor=east] {$b_1$}
    
    (mux.input 4) ++(-1,-.2) node[or gate, inputs=nn] (or_ab) {}
    (or_b.output) ++(.2,0) node[contact] {}
    (or_b.output) ++(.2,0) |- (or_ab.input 2)
    
    (mux.input 2) ++(-1.7,0) node[contact] {}
    (mux.input 2) ++(-1.7,0) |- (or_ab.input 1)
  ;
  \draw[<-] (mux.input 1) -- ++(-.5,0) node[anchor=east] {$0$};
  \draw[->] (or_a.output) -- ++(.2,0) |- (mux.input 2);
  \draw[->] (or_b.output) -- ++(.2,0) |- (mux.input 3);
  \draw[->] (or_ab.output) -- ++(.2,0) |- (mux.input 4);
  \draw[ultra thick] (mux.select) -- ++(0,-.5) node[anchor=north] {$b_3a_3$};
  \draw[->] (mux.output) -- ++(.5,0) node[anchor=west] {$ERR$};
\end{tikzpicture} \end{center}

\subparagraph{Apartat D}

Hem de tenir en compte que desplaçar (emplenant amb zeros) un nombre codificat
en binari natural $n$ bits cap a la dreta equival a multiplicar per $2^n$
el valor codificat. Llavors, el bloc \textsf{B1} es pot descriure com:
%
\begin{equation*}
  Y = \left(B \cdot 2^2\right) + B = 5B
\end{equation*}
%
La suma no sortirà mai de rang per a cap de les entrades possibles, per tant
l'expressió descriu perfectament el comportament del bloc.

Anàlogament, per a \textsf{B2} tenim:
%
\begin{equation*}
  Z = \left(Y \cdot 2^1\right) + A = 2Y + A
\end{equation*}

Per tant, els blocs \textsf{B1} i \textsf{B2} conjuntament efectuen l'operació:
%
\begin{equation*}
  Z = 10B + A
\end{equation*}
%
En altres paraules: es retorna a $Z$ el valor, en binari natural,
que representen $b_3...b_0\,a_3...a_0$ codificats en BCD.

Estem davant un \emph{convertidor de BCD natural de dues xifres a binari natural}.

\subparagraph{Apartat E}

Al primer buit:

\begin{minted}{vhdl}
signal P : std_logic_vector(7 downto 0);
\end{minted}

Al segon buit:

\begin{minted}{vhdl}
P <= "1010" * B;
Z <= P(6 downto 0) + ("000" & A);
\end{minted}

\finishpage

\end{document}
